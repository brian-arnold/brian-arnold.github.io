% Options for packages loaded elsewhere
\PassOptionsToPackage{unicode}{hyperref}
\PassOptionsToPackage{hyphens}{url}
%
\documentclass[
]{article}
\usepackage{lmodern}
\usepackage{amssymb,amsmath}
\usepackage{ifxetex,ifluatex}
\ifnum 0\ifxetex 1\fi\ifluatex 1\fi=0 % if pdftex
  \usepackage[T1]{fontenc}
  \usepackage[utf8]{inputenc}
  \usepackage{textcomp} % provide euro and other symbols
\else % if luatex or xetex
  \usepackage{unicode-math}
  \defaultfontfeatures{Scale=MatchLowercase}
  \defaultfontfeatures[\rmfamily]{Ligatures=TeX,Scale=1}
\fi
% Use upquote if available, for straight quotes in verbatim environments
\IfFileExists{upquote.sty}{\usepackage{upquote}}{}
\IfFileExists{microtype.sty}{% use microtype if available
  \usepackage[]{microtype}
  \UseMicrotypeSet[protrusion]{basicmath} % disable protrusion for tt fonts
}{}
\makeatletter
\@ifundefined{KOMAClassName}{% if non-KOMA class
  \IfFileExists{parskip.sty}{%
    \usepackage{parskip}
  }{% else
    \setlength{\parindent}{0pt}
    \setlength{\parskip}{6pt plus 2pt minus 1pt}}
}{% if KOMA class
  \KOMAoptions{parskip=half}}
\makeatother
\usepackage{xcolor}
\IfFileExists{xurl.sty}{\usepackage{xurl}}{} % add URL line breaks if available
\IfFileExists{bookmark.sty}{\usepackage{bookmark}}{\usepackage{hyperref}}
\hypersetup{
  pdftitle={Untitled},
  hidelinks,
  pdfcreator={LaTeX via pandoc}}
\urlstyle{same} % disable monospaced font for URLs
\usepackage[margin=1in]{geometry}
\usepackage{color}
\usepackage{fancyvrb}
\newcommand{\VerbBar}{|}
\newcommand{\VERB}{\Verb[commandchars=\\\{\}]}
\DefineVerbatimEnvironment{Highlighting}{Verbatim}{commandchars=\\\{\}}
% Add ',fontsize=\small' for more characters per line
\usepackage{framed}
\definecolor{shadecolor}{RGB}{248,248,248}
\newenvironment{Shaded}{\begin{snugshade}}{\end{snugshade}}
\newcommand{\AlertTok}[1]{\textcolor[rgb]{0.94,0.16,0.16}{#1}}
\newcommand{\AnnotationTok}[1]{\textcolor[rgb]{0.56,0.35,0.01}{\textbf{\textit{#1}}}}
\newcommand{\AttributeTok}[1]{\textcolor[rgb]{0.77,0.63,0.00}{#1}}
\newcommand{\BaseNTok}[1]{\textcolor[rgb]{0.00,0.00,0.81}{#1}}
\newcommand{\BuiltInTok}[1]{#1}
\newcommand{\CharTok}[1]{\textcolor[rgb]{0.31,0.60,0.02}{#1}}
\newcommand{\CommentTok}[1]{\textcolor[rgb]{0.56,0.35,0.01}{\textit{#1}}}
\newcommand{\CommentVarTok}[1]{\textcolor[rgb]{0.56,0.35,0.01}{\textbf{\textit{#1}}}}
\newcommand{\ConstantTok}[1]{\textcolor[rgb]{0.00,0.00,0.00}{#1}}
\newcommand{\ControlFlowTok}[1]{\textcolor[rgb]{0.13,0.29,0.53}{\textbf{#1}}}
\newcommand{\DataTypeTok}[1]{\textcolor[rgb]{0.13,0.29,0.53}{#1}}
\newcommand{\DecValTok}[1]{\textcolor[rgb]{0.00,0.00,0.81}{#1}}
\newcommand{\DocumentationTok}[1]{\textcolor[rgb]{0.56,0.35,0.01}{\textbf{\textit{#1}}}}
\newcommand{\ErrorTok}[1]{\textcolor[rgb]{0.64,0.00,0.00}{\textbf{#1}}}
\newcommand{\ExtensionTok}[1]{#1}
\newcommand{\FloatTok}[1]{\textcolor[rgb]{0.00,0.00,0.81}{#1}}
\newcommand{\FunctionTok}[1]{\textcolor[rgb]{0.00,0.00,0.00}{#1}}
\newcommand{\ImportTok}[1]{#1}
\newcommand{\InformationTok}[1]{\textcolor[rgb]{0.56,0.35,0.01}{\textbf{\textit{#1}}}}
\newcommand{\KeywordTok}[1]{\textcolor[rgb]{0.13,0.29,0.53}{\textbf{#1}}}
\newcommand{\NormalTok}[1]{#1}
\newcommand{\OperatorTok}[1]{\textcolor[rgb]{0.81,0.36,0.00}{\textbf{#1}}}
\newcommand{\OtherTok}[1]{\textcolor[rgb]{0.56,0.35,0.01}{#1}}
\newcommand{\PreprocessorTok}[1]{\textcolor[rgb]{0.56,0.35,0.01}{\textit{#1}}}
\newcommand{\RegionMarkerTok}[1]{#1}
\newcommand{\SpecialCharTok}[1]{\textcolor[rgb]{0.00,0.00,0.00}{#1}}
\newcommand{\SpecialStringTok}[1]{\textcolor[rgb]{0.31,0.60,0.02}{#1}}
\newcommand{\StringTok}[1]{\textcolor[rgb]{0.31,0.60,0.02}{#1}}
\newcommand{\VariableTok}[1]{\textcolor[rgb]{0.00,0.00,0.00}{#1}}
\newcommand{\VerbatimStringTok}[1]{\textcolor[rgb]{0.31,0.60,0.02}{#1}}
\newcommand{\WarningTok}[1]{\textcolor[rgb]{0.56,0.35,0.01}{\textbf{\textit{#1}}}}
\usepackage{graphicx,grffile}
\makeatletter
\def\maxwidth{\ifdim\Gin@nat@width>\linewidth\linewidth\else\Gin@nat@width\fi}
\def\maxheight{\ifdim\Gin@nat@height>\textheight\textheight\else\Gin@nat@height\fi}
\makeatother
% Scale images if necessary, so that they will not overflow the page
% margins by default, and it is still possible to overwrite the defaults
% using explicit options in \includegraphics[width, height, ...]{}
\setkeys{Gin}{width=\maxwidth,height=\maxheight,keepaspectratio}
% Set default figure placement to htbp
\makeatletter
\def\fps@figure{htbp}
\makeatother
\setlength{\emergencystretch}{3em} % prevent overfull lines
\providecommand{\tightlist}{%
  \setlength{\itemsep}{0pt}\setlength{\parskip}{0pt}}
\setcounter{secnumdepth}{-\maxdimen} % remove section numbering

\title{Untitled}
\author{}
\date{\vspace{-2.5em}}

\begin{document}
\maketitle

Loading in libraries. I will use `mvtnorm' to simulate two
noramlly-distributed variables that are imperfectly correlated.

\begin{Shaded}
\begin{Highlighting}[]
\KeywordTok{library}\NormalTok{(mvtnorm)}
\KeywordTok{library}\NormalTok{(scales)}
\KeywordTok{library}\NormalTok{(tidyverse)}
\end{Highlighting}
\end{Shaded}

\begin{verbatim}
## -- Attaching packages --------------------------------------------------------------------------------------- tidyverse 1.3.0 --
\end{verbatim}

\begin{verbatim}
## v ggplot2 3.3.2     v purrr   0.3.4
## v tibble  3.0.3     v dplyr   1.0.0
## v tidyr   1.1.0     v stringr 1.4.0
## v readr   1.3.1     v forcats 0.5.0
\end{verbatim}

\begin{verbatim}
## -- Conflicts ------------------------------------------------------------------------------------------ tidyverse_conflicts() --
## x readr::col_factor() masks scales::col_factor()
## x purrr::discard()    masks scales::discard()
## x dplyr::filter()     masks stats::filter()
## x dplyr::lag()        masks stats::lag()
\end{verbatim}

Let's have our two variables have a correlation of 0.9, 0.5, and 0.1 to
look at the effect. We will store these values in a vector that we will
iterate across.

\begin{Shaded}
\begin{Highlighting}[]
\NormalTok{covars <-}\StringTok{ }\KeywordTok{c}\NormalTok{(}\FloatTok{0.8}\NormalTok{, }\FloatTok{0.2}\NormalTok{)}
\end{Highlighting}
\end{Shaded}

\begin{Shaded}
\begin{Highlighting}[]
\KeywordTok{dev.new}\NormalTok{(}\DataTypeTok{width=}\DecValTok{10}\NormalTok{, }\DataTypeTok{height=}\DecValTok{7}\NormalTok{, }\DataTypeTok{noRStudioGD=}\NormalTok{T)}
\KeywordTok{par}\NormalTok{(}\DataTypeTok{bty=}\StringTok{"n"}\NormalTok{, }\DataTypeTok{mfrow=}\KeywordTok{c}\NormalTok{(}\DecValTok{2}\NormalTok{,}\DecValTok{2}\NormalTok{)) }\CommentTok{# make a plot with no box type with 6 panels: 3 rows and 2 columns}
\ControlFlowTok{for}\NormalTok{ (covar }\ControlFlowTok{in}\NormalTok{ covars)\{}
  
\NormalTok{  covar_mat <-}\StringTok{ }\KeywordTok{rbind}\NormalTok{(}\KeywordTok{c}\NormalTok{(}\DecValTok{1}\NormalTok{,covar), }\KeywordTok{c}\NormalTok{(covar,}\DecValTok{1}\NormalTok{))    }\CommentTok{# create covariance matrix}

  \CommentTok{# create table of draws from a bivariate normal distribution}
\NormalTok{  d <-}\StringTok{ }\KeywordTok{rmvnorm}\NormalTok{(}\DataTypeTok{n=}\DecValTok{10000}\NormalTok{, }\DataTypeTok{mean =} \KeywordTok{rep}\NormalTok{(}\DecValTok{0}\NormalTok{,}\DecValTok{2}\NormalTok{), }\DataTypeTok{sigma=}\NormalTok{covar_mat) }\OperatorTok
\StringTok{    }\KeywordTok{as_tibble}\NormalTok{() }\OperatorTok\StringTok{ }\KeywordTok{rename}\NormalTok{(}\StringTok{"x"}\NormalTok{ =}\StringTok{ }\NormalTok{V1, }\StringTok{"y"}\NormalTok{ =}\StringTok{ }\NormalTok{V2)}
  
  \CommentTok{# create another table that selects only extreme values for y }
\NormalTok{  d_red <-}\StringTok{ }\NormalTok{d }\OperatorTok\StringTok{ }
\StringTok{    }\KeywordTok{filter}\NormalTok{(y }\OperatorTok{>=}\StringTok{ }\DecValTok{2}\NormalTok{)}
    
  \CommentTok{# make a plot}
  \KeywordTok{plot}\NormalTok{(d}\OperatorTok{$}\NormalTok{x, d}\OperatorTok{$}\NormalTok{y, }
     \DataTypeTok{pch=}\DecValTok{16}\NormalTok{, }
     \DataTypeTok{col=}\KeywordTok{alpha}\NormalTok{(}\StringTok{"black"}\NormalTok{, }\FloatTok{0.1}\NormalTok{),}
     \DataTypeTok{main=}\KeywordTok{paste}\NormalTok{(}\KeywordTok{c}\NormalTok{(}\StringTok{"covariance = "}\NormalTok{, covar), }\DataTypeTok{sep=}\StringTok{""}\NormalTok{),}
     \DataTypeTok{xlab=}\StringTok{"x"}\NormalTok{,}
     \DataTypeTok{ylab=}\StringTok{"y"}\NormalTok{,}
     \DataTypeTok{axes=}\NormalTok{F, }
     \DataTypeTok{ylim=}\KeywordTok{c}\NormalTok{(}\OperatorTok{-}\FloatTok{3.5}\NormalTok{, }\FloatTok{3.5}\NormalTok{),}
     \DataTypeTok{xlim=}\KeywordTok{c}\NormalTok{(}\OperatorTok{-}\FloatTok{3.5}\NormalTok{, }\FloatTok{3.5}\NormalTok{))}
  \KeywordTok{points}\NormalTok{(d_red}\OperatorTok{$}\NormalTok{x, }
\NormalTok{    d_red}\OperatorTok{$}\NormalTok{y, }
    \DataTypeTok{col=}\KeywordTok{alpha}\NormalTok{(}\StringTok{"red"}\NormalTok{, }\FloatTok{0.3}\NormalTok{), }
    \DataTypeTok{pch=}\DecValTok{16}\NormalTok{)}
  \KeywordTok{axis}\NormalTok{(}\DataTypeTok{side=}\DecValTok{1}\NormalTok{,  }\DataTypeTok{labels=}\NormalTok{T, }\DataTypeTok{at=}\KeywordTok{c}\NormalTok{(}\OperatorTok{-}\DecValTok{3}\NormalTok{,}\OperatorTok{-}\DecValTok{2}\NormalTok{,}\OperatorTok{-}\DecValTok{1}\NormalTok{,}\OperatorTok{-}\DecValTok{1}\NormalTok{,}\DecValTok{0}\NormalTok{,}\DecValTok{1}\NormalTok{,}\DecValTok{2}\NormalTok{,}\DecValTok{3}\NormalTok{))}
  \KeywordTok{axis}\NormalTok{(}\DataTypeTok{side=}\DecValTok{2}\NormalTok{, }\DataTypeTok{labels=}\NormalTok{T, }\DataTypeTok{at=}\KeywordTok{c}\NormalTok{(}\OperatorTok{-}\DecValTok{3}\NormalTok{,}\OperatorTok{-}\DecValTok{2}\NormalTok{,}\OperatorTok{-}\DecValTok{1}\NormalTok{,}\OperatorTok{-}\DecValTok{1}\NormalTok{,}\DecValTok{0}\NormalTok{,}\DecValTok{1}\NormalTok{,}\DecValTok{2}\NormalTok{,}\DecValTok{3}\NormalTok{))}
  \KeywordTok{abline}\NormalTok{(}\DataTypeTok{h=}\KeywordTok{c}\NormalTok{(}\DecValTok{2}\NormalTok{,}\DecValTok{3}\NormalTok{), }\DataTypeTok{lty=}\DecValTok{3}\NormalTok{, }\DataTypeTok{col=}\StringTok{"blue"}\NormalTok{, }\DataTypeTok{lwd=}\DecValTok{2}\NormalTok{)}
  \KeywordTok{abline}\NormalTok{(}\DataTypeTok{v=}\KeywordTok{c}\NormalTok{(}\DecValTok{2}\NormalTok{,}\DecValTok{3}\NormalTok{), }\DataTypeTok{lty=}\DecValTok{3}\NormalTok{, }\DataTypeTok{col=}\StringTok{"blue"}\NormalTok{, }\DataTypeTok{lwd=}\DecValTok{2}\NormalTok{)}
  \KeywordTok{hist}\NormalTok{(d_red}\OperatorTok{$}\NormalTok{x, }\DataTypeTok{col=}\KeywordTok{alpha}\NormalTok{(}\StringTok{"red"}\NormalTok{, }\FloatTok{0.5}\NormalTok{), }\DataTypeTok{xlab=}\StringTok{"x"}\NormalTok{, }\DataTypeTok{freq=}\NormalTok{F, }\DataTypeTok{main=}\StringTok{""}\NormalTok{, }\DataTypeTok{breaks=}\DecValTok{20}\NormalTok{)}
  \CommentTok{#abline(v=c(2,3), lty=3, col="blue", lwd=2)}
\NormalTok{\}}
\end{Highlighting}
\end{Shaded}

\begin{verbatim}
## Warning: The `x` argument of `as_tibble.matrix()` must have unique column names if `.name_repair` is omitted as of tibble 2.0.0.
## Using compatibility `.name_repair`.
## This warning is displayed once every 8 hours.
## Call `lifecycle::last_warnings()` to see where this warning was generated.
\end{verbatim}

\end{document}
